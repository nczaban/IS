% !TEX root = ../IS.tex
\chapter{Introduction}
As the computing power of personal computers has increased, the depth and variety of interactive computer games has similarly increased. The earliest video games required two human players, and did not have any digital opponents. Games like Space Invaders and Pursuit introduced enemies which moved in set patterns; as time progressed, the intelligence of these enemies was improved with more intricate patterns (as in the space shooter Galaxian) or in more heavily-scripted responses to inputs (as in the first-person shooter Half-Life) \cite{schw04}. In more recent titles, AI systems begin to show more real intelligence, instead of preset patterns determined by a game designer.\\

In many role-playing games (RPGs), the heroes live in fantastical worlds and often have the ability to quickly heal themselves of their wounds with a quick potion or magic spell. While the player-controlled characters have these abilities, their AI-controlled opponents frequently do not. In rare cases where these enemies can heal themselves, this ability is often relegated to one-time uses determined by the game designers. This thesis concerns itself with designing an AI which can decide when it will use a healing ability, and use game theory to analyze the resulting game.